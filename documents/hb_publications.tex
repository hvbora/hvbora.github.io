
% --- Set document class and font size ---

\documentclass[a4paper, 11pt]{article}





% --- Package imports ---

\usepackage{hyperref, enumitem, longtable, fontawesome, amsmath, array, graphicx}





% --- Page layout settings ---

% Set page margins
\usepackage[left=0.7in, right=0.8in, bottom=.8in, top=0.8in, headsep=0in, footskip=.2in]{geometry}

% Set line spacing
\renewcommand{\baselinestretch}{1.2}

% --- Page formatting settings ---

% Set link colors
\usepackage[dvipsnames]{xcolor}
\hypersetup{colorlinks=true, linkcolor=MidnightBlue, urlcolor=MidnightBlue}

% Set font to Libertine, including math support
\usepackage{libertine}
\usepackage[libertine]{newtxmath}

% Remove page numbering
\pagenumbering{gobble}

% Define font size and color for section headings
\newcommand{\headingfont}{\Large\color{Peach}}











% --- Publications section settings ---

% Note: each section of this table (Education, Awards, Publications etc.) is 
% stored in a two-column table. The left-hand column is narrow (1 inch) and is 
% meant to store dates. The right-hand column is wide (5.2 inches) and stores 
% the main text.  Sections in which each entry might have multiple lines 
% (e.g., Education) are stored in a 'SectionTable' environment). Sections in 
% which each entry might just have one line are stored in a 'SectionTableSingleSpace'
% environment. The only difference between the two environments is the line 
% spacing between each entry. Both environments take one argument, which is the
% title of the section. See main document for how these environments are used.

% Define settings for left-hand column in which dates are printed
\newcolumntype{R}{>{\raggedleft}p{1in}}

% Define 'SectionTable' environment
\newenvironment{SectionTable}[1]{
	\renewcommand*{\arraystretch}{1.7}
	\setlength{\tabcolsep}{10pt}
	\begin{longtable}{Rp{5.2in}} & #1 \\}
{\end{longtable}\vspace{-.3cm}}

% Define 'SectionTableSingleSpace' environment
\newenvironment{SectionTableSingleSpace}[1]{
	\renewcommand*{\arraystretch}{1.2}
	\setlength{\tabcolsep}{10pt}
	\begin{longtable}{Rp{5.2in}} & #1 \\[0.6em]}
{\end{longtable}\vspace{-.3cm}}















% --- Document starts here ---

\begin{document}

% --- Name and contact information ---

\begin{SectionTable}{\Huge \color{Peach} Himanshu Bora}
& 
\faMapMarker \ \ DHSK College $\;\boldsymbol{\cdot}\;$ Dibrugarh, Assam, India \newline
\faEnvelope \  \textit{hvbora@gmail.com} \ \ \ \  \faPhone \ +91 84866 09233 \\
%\faGlobe \ www.website.com %
&
\href{https://en.wikibooks.org/wiki/LaTeX/Hyperlinks}{ArXiv} 
$\;\boldsymbol{\cdot}\;$
\href{https://en.wikibooks.org/wiki/LaTeX/Hyperlinks}{ORCID} 
$\;\boldsymbol{\cdot}\;$
\href{https://en.wikibooks.org/wiki/LaTeX/Hyperlinks}{Google Scholar}
$\;\boldsymbol{\cdot}\;$
\href{https://en.wikibooks.org/wiki/LaTeX/Hyperlinks}{InspireHEP}
\end{SectionTable}





% --- Section: Publications ---

\begin{SectionTable}{\headingfont Publications} 
2021 & 
\textbf{Title of your most recent research paper} \newline
First author, second author, third author, fourth author. \newline
\textit{Journal of something or the other}
(\href{https://en.wikibooks.org/wiki/LaTeX/Hyperlinks}{DOI: }) \\

\end{SectionTable}


% --- Section: Conference Proceedings ---

\begin{SectionTable}{\headingfont Conference Proceedings} 
2021 & 
\textbf{Title of your most recent research paper} \newline
First author, second author, third author, fourth author. \newline
\textit{Journal of something or the other}
(\href{https://en.wikibooks.org/wiki/LaTeX/Hyperlinks}{DOI: }) \\

\end{SectionTable}





\end{document}





