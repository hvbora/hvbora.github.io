
% --- Set document class and font size ---

\documentclass[a4paper, 11pt]{article}





% --- Package imports ---

\usepackage{hyperref, enumitem, longtable, fontawesome, amsmath, array, graphicx}





% --- Page layout settings ---

% Set page margins
\usepackage[left=0.7in, right=0.8in, bottom=.8in, top=0.8in, headsep=0in, footskip=.2in]{geometry}

% Set line spacing
\renewcommand{\baselinestretch}{1.2}

% --- Page formatting settings ---

% Set link colors
\usepackage[dvipsnames]{xcolor}
\hypersetup{colorlinks=true, linkcolor=MidnightBlue, urlcolor=MidnightBlue}

% Set font to Libertine, including math support
\usepackage{libertine}
\usepackage[libertine]{newtxmath}

% Remove page numbering
\pagenumbering{gobble}

% Define font size and color for section headings
\newcommand{\headingfont}{\Large\color{Bittersweet}}




% --- CV section settings ---

% Note: each section of this table (Education, Awards, Publications etc.) is 
% stored in a two-column table. The left-hand column is narrow (1 inch) and is 
% meant to store dates. The right-hand column is wide (5.2 inches) and stores 
% the main text.  Sections in which each entry might have multiple lines 
% (e.g., Education) are stored in a 'SectionTable' environment). Sections in 
% which each entry might just have one line are stored in a 'SectionTableSingleSpace'
% environment. The only difference between the two environments is the line 
% spacing between each entry. Both environments take one argument, which is the
% title of the section. See main document for how these environments are used.

% Define settings for left-hand column in which dates are printed
\newcolumntype{R}{>{\raggedleft}p{1in}}

% Define 'SectionTable' environment
\newenvironment{SectionTable}[1]{
	\renewcommand*{\arraystretch}{1.7}
	\setlength{\tabcolsep}{10pt}
	\begin{longtable}{Rp{5.2in}} & #1 \\}
{\end{longtable}\vspace{-.3cm}}

% Define 'SectionTableSingleSpace' environment
\newenvironment{SectionTableSingleSpace}[1]{
	\renewcommand*{\arraystretch}{1.2}
	\setlength{\tabcolsep}{10pt}
	\begin{longtable}{Rp{5.2in}} & #1 \\[0.6em]}
{\end{longtable}\vspace{-.3cm}}















% --- Document starts here ---

\begin{document}

% --- Name and contact information ---

\begin{SectionTable}{\Huge \color{Bittersweet} Himanshu Bora}
&{\large Assistant Professor $\;\boldsymbol{\cdot}\;$ Department of Physics} \newline
Kamrup College, Chamata $\;\boldsymbol{\cdot}\;$ Nalbari, Assam, India \newline
\faEnvelope \ \ \textit{hb@kamrupcollege.ac.in} \ \ \ \  \faPhone \ +91 84866 09233 \ \ \ \ 
\faGlobe \ \ \href{https://hvbora.github.io}{\textit{hvbora.github.io}}  \\

&
\href{https://orcid.org/0000-0001-9694-4441}{ORCID} 
$\;\boldsymbol{\cdot}\;$
\href{https://vidwan.inflibnet.ac.in/profile/593335}{Vidwan}
$\;\boldsymbol{\cdot}\;$
\href{https://scholar.google.com/citations?user=2mxLUZ0AAAAJ&hl=en}{Google Scholar}
$\;\boldsymbol{\cdot}\;$
%\href{https://en.wikibooks.org/wiki/LaTeX/Hyperlinks}{InspireHEP}
%$\;\boldsymbol{\cdot}\;$ 
\href{https://arxiv.org/a/0000-0001-9694-4441.html}{ArXiv} 
\\
		
&
{\textbf {Personal Information}} \newline
Date of birth \ \ : \ December 21, 1996 \newline
%Sex \ \ \ \ \ \ \ \ \ \ \ \ \ \ \ \ \ : \ Male \newline
Citizenship \ \ \ \ : \ Indian \newline
Languages \ \ \ \ \ : \ English (\textit{Full Professional Proficiency})  $\;\boldsymbol{\cdot}\;$  Assamese (\textit{Native})\\
\end{SectionTable}



% --- Section: Education ---
\begin{SectionTable}{\headingfont Education}

2024 -- present & 
\textbf{Ph.D.}  \newline
Department of Physics \newline
Bhattadev University -- Bajali, Assam, India \newline 
Title: \textit{T.B.D.} \newline
Guide: \href{https://scholar.google.com/citations?user=_8gOgy0AAAAJ&hl=en }{\textit{Dr. Debajyoti Dutta}}  \\

2015 -- 2020 & 
\textbf{Integrated M.Sc. in Physics} \newline
Department of Physics \newline
Dibrugarh University -- Dibrugarh, Assam, India \newline 
Specialization: High Energy Physics \\
%CGPA: 8.27/10.00 (87.7\%)\newline
%6th Semester Project: A Primitive Study on Information Geometry \\


2012 -- 2015 & 
\textbf{HSSLC (12th Standard)} \ Stream: Science \newline
Board: Assam Higher Secondary Education Council (AHSEC) \newline 
School: Sarupathar H.S. School -- Sarupathar, Assam, India \\
%Percentage: 90.40\% \\

2012 &
\textbf{HSLC (10th Standard)} \newline
Board: Board of Secondary Education, Assam (SEBA) \newline 
School: Dayanand Vidya Niketan -- Sarupathar, Assam, India \\
%Percentage: 90.00\%  \\

\end{SectionTable}




% --- Section: Professional Experience ---

\begin{SectionTable}{\headingfont Professional Expereince}

Sep 2022 \newline -- present & 
\textbf{Assistant Professor} \newline
Department of Physics\newline
Kamrup College -- Chamata, Assam, India \\
%Courses Taught: Mathematical Physics, Mechanics, Electrodynamics, Programming in C, Numerical Methods and Computational Physics \\


Nov 2021 \newline -- Apr 2022 & 
\textbf{Lecturer} (\textit{contractual}) \newline
Department of Physics\newline
Dibrugarh Hanumanbax Surajmall Kanoi College -- Dibrugarh, Assam, India \\
%Courses Taught: Mathematical Physics, Computational Physics, Quantum Mechanics \\


\end{SectionTable}

% --- Section: Teaching ---

\begin{SectionTable}{\headingfont Courses Taught (\textit{selected})}
%Classical Dynamics $\;\boldsymbol{\cdot}\;$ 
%\href{https://hvbora.github.io}{\textit{Lecture notes}} $\;\boldsymbol{\cdot}\;$ 
%\href{https://hvbora.github.io}{\textit{Codes}} \newline
&
Mathematical Physics I,
Mathematical Physics II,
Mathematical Physics III, \newline
Classical Mechanics, 
Quantum Mechanics and Applications, \newline
Basic Programming in C,
Computational Physics Skill \\

\end{SectionTable}


% --- Section: Research interests ---

\begin{SectionTable}{\headingfont Research Interests}
& Neutrino Oscillations, Beyond Standard Model Physics
\end{SectionTable} 



% --- Section: Journal Publications ---

%\begin{SectionTable}{\headingfont Journal Articles} 
%2021 & 
%\textbf{Title of your most recent research paper} \newline
%First author, second author, third author, fourth author. \newline
%\textit{Journal of something or the other}
%(\href{https://en.wikibooks.org/wiki/LaTeX/Hyperlinks}{DOI: }) \\
	
%\end{SectionTable}
	
	
% --- Section: Conference Proceedings ---
	
%\begin{SectionTable}{\headingfont Conference Proceedings} 
%2021 & 
%\textbf{Title of your most recent research paper} \newline
%First author, second author, third author, fourth author. \newline
%\textit{Journal of something or the other}
%(\href{https://en.wikibooks.org/wiki/LaTeX/Hyperlinks}{DOI: }) \\
	
%\end{SectionTable}


% --- Section: Books and Book Chapters ---
	
%\begin{SectionTable}{\headingfont Books and Book Chapters} 
%2021 & 
%\textbf{Title of your most recent research paper} \newline
%First author, second author, third author, fourth author. \newline
%\textit{Journal of something or the other}
%(\href{https://en.wikibooks.org/wiki/LaTeX/Hyperlinks}{DOI: }) \\
%\end{SectionTable}
	

% --- Section: Talks and tutorials ---

%\begin{SectionTable}{\headingfont Talks and Tutorials}
%Month Year &
%Title of your most recent presentation \newline
%\textit{Name of conference, workshop, seminar, etc., or a description} \\

%Month Year &
%Title of your second most recent presentation \newline
%\textit{Name of conference, workshop, seminar, etc., or a description} \\

%Month Year &
%Title of your third most recent presentation \newline
%\textit{Name of conference, workshop, seminar, etc., or a description} %\\
%\end{SectionTable}


% --- Section: Research experience ---

\begin{SectionTable}{\headingfont Research Experience}

Oct 2020 \newline -- Jul 2021 &
Dibrugarh University -- Dibrugarh, Assam, India \newline
\textbf{Topic:} Geometrothermodynamics of Some Black Hole Systems \newline
\textbf{Guide:} Dr. Prabwal Jyoti Phukon, Dibrugarh University, Dibrugarh, India  \\

Jun 2017 \newline -- Aug 2017 &
Bharathiar University -- Coimbatore, Tamil Nadu, India (under FAST-SF program) \newline
\textbf{Title:} A Transformation Method for Construction of Exactly Solvable Potentials \newline
\textbf{Guide:} Prof. S. Saravanan, Bharathiar University, Coimbatore, India
\\
\end{SectionTable}


% --- Section: Projects and Grants ---

%\begin{SectionTable}{\headingfont Projects and Grants}
%June 2017 --August 2017 &
%\textbf{PI/Co-PI:}  Project Name \newline
%\textbf{Funding Agency:} Description \newline
%\href{https://hvbora.github.io}{\textit{Link}} 
%\\		
%\end{SectionTable}

% --- Section: Mentorship experience ---

%\begin{SectionTable}{\headingfont Mentorship Experience}

%October 2020 --July 2021 &
%Dibrugarh University -- Dibrugarh, India \newline
%\textbf{Topic:} Geometrothermodynamics of Some Black Hole Systems \newline
%\textbf{Guide:} Dr. Prabwal Jyoti Phukon, Dibrugarh University, Dibrugarh, India  \\
%\end{SectionTable}


% --- Section: Conferences, Workshops and schools, etc ---

\begin{SectionTable}{\headingfont Schools, Workshops and Conferences Attended}

Dec 9 -- Dec 13 2024 &
GIAN course on \textit{Standard Model Effective Field Theories and Applications to Higgs, Neutrinos and Dark Matter} 
by Department of Physics and Centre of Educational Technology, Indian Institute of Technology Guwahati
\\

Jan 6 -- Jan 10 2024 &
\textit{Pedagogic Workshop on Astronomy, Astrophysics and Cosmology: A Faculty Enrichment Programme} 
by Department of Physics, Gauhati University and Inter University Center for Astronomy and Astrophysics (IUCAA-TLC and IUCAA-ACE)
\\

Dec 12 - Dec 22 2023 &
\textit{Radio Astronomy Winter School-2023} 
by Inter University Center for Astronomy and Astrophysics (IUCAA)  and National Center for Radio Astrophysics (NCRA-TIFR) (\textit{participated as mentor to student participants})
\\

Jan 6 -- Mar 4 2023 &
\textit{Winter School on Deep Learning: From Perceptrons to Diffusion Models} 
by Electronics and Communication Sciences Unit, Indian Statistical Institute Kolkata (online mode)
\\

Jul 25 -- Jul 29 2022 & 
\textit{Online Workshop on Artificial Intelligence and High Performance Computing} 
by National Institute of Technology Mizoram \& Indian Institute of Technology Kharagpur
\\
	 
May 26 -May 30 2022 &
\textit{Workshop on Space and Atmospheric Sciences' Research Tools} 
by Department of Physics, Dibrugarh University
\\

%May 17 -- 20, 2022 & 
%\textit{Online Workshop on High Performance Computing for Computational Fluid Dynamics Applications} by Indian Institute of Technology, Bombay and Center for Development of Advanced Computing (C-DAC)
% \\

%October 29 -- 30, 2021 & 
%\textit{Online Workshop on High Performance Computing and AI for Computational Biology} by Indian Institute of Technology, Kharagpur and Tezpur University, Tezpur
% \\
%October 21 -- 22, 2021 & 
%\textit{Online Workshop on High Performance Computing in Engineering} by Indian Institute of Technology, Kharagpur and Ansys
% \\
 
Sep 20 -- Sep 23 2021 & 
\textit{Online Workshop on High Performance Computing for Astrophysics and Astronomy} 
by Square Kilometer Array- India Consort. \& Indian Institute of Technology Kharagpur
 \\

May 10 - Jun 11 2021 &
\textit{Introductory Summer School in Astronomy and Astrophysics} 
by Inter University Center for Astronomy and Astrophysics (IUCAA) (\textit{online mode})
\\

Aug 17 - Aug 31 2021 &
\textit{Workshop on Particle Physics, 2020} 
by Department of Physics, Assam Don Bosco University (\textit{online mode})
\\

Jan 7 -- Jan 11 2019 &
\textit{Workshop on Celestial Mechanics and Dynamical Astronomy} 
by Department of Mathematics, Central University of Rajasthan
\\

Oct 7 -- Oct 13 2018 &
\textit{Workshop on Gravitation and Gravitational Waves} 
by Department of Physics, Assam University, Silchar
\\

Feb 22 -- Mar 3 2017 &
\textit{ARIES Training School in Observational Astronomy-2017} 
by Aryabhatta Research Institute of Observational Sciences, Nainital
\\

Jun 21 -- Jul 12 2016 &
\textit{CCSU Astronomy and Astrophysics Summer School-2016} 
by Department of Physics, Cotton College State University
\\

Nov 2 -- Nov 5 2015 &
\textit{National Conference on Current Issues in Cosmology, Astrophysics and High Energy Physics} 
by Dibrugarh University
\\
\end{SectionTable}


% --- Section: Honours, Awards, Scholarships, etc ---

\begin{SectionTable}{\headingfont Honors, Awards and Scholarships}
Jun 2021 & 
Awarded \textbf{INSPIRE Fellowship 2020} provisionally for doctoral research by Department of Science and Technology, Government of India \textit{(provisional award not availed)}
\\
	
May 2021 &
Awarded \textbf{Junior Research Fellowship} by Council of Scientific and
Industrial Research, India through \textbf{Joint CSIR-UGC NET Examination, June 2020}
(held in November 2020) \\ %\textit{All India Rank-149 (99.51 percentile)} \\
	
May 2021 &
Obtained eligibility for \textbf{Lectureship / Assistant Professor} in Physical Sciences through \textbf{Joint CSIR-UGC NET Examination, June 2020}
(held in November 2020), \textit{All India Rank-149 (99.51 percentile)} \\
	
Feb 2020 &
Qualified \textbf{Graduate Aptitude Test in Engineering-2020 (GATE-2020)} in Physics organised jointly by IISc/IITs (held in February 2020) \\
	
2017 &
Awarded \textbf{Focus Area Science Technology Summer Fellowship (FAST-SF)} jointly by Indian Academy of Sciences, Bengaluru; Indian National Science Academy, New Delhi and The National Academy of Sciences, Prayagraj \\
	
2015-2020 (\textit{duration})&
Awarded \textbf{INSPIRE-SHE} Scholarship by Department of Science and Technology, Government of India (\textit{awarded to top 1\% students in India in 10+2, pursuing basic sciences}) \\
	
2012 &
Awarded \textbf{Anundoram Borooah Award-2012} with distinction by Government of Assam (\textit{awarded to First Division holders in HSLC Examination, 2012 conducted by SEBA, Assam}) \\
	
\end{SectionTable}

% --- Section: Professional Society memberships ---

\begin{SectionTable}{\headingfont Professional Memberships}
Jan 2025 --  &
Physics Academy of North East (\textit{lifetime membership}) \\
Feb 2023 --  &
Assam Physical Society (\textit{lifetime membership}) \\

%\textit{Short description or conferences you attended.} \\

\end{SectionTable}


% --- Section: Technical Skills ---
\begin{SectionTable}{\headingfont Technical Skills}
& \textbf{Languages} \newline
Proficient in C, Python \newline
Familiar with C++, Fortran \\

& \textbf{Others} \newline
Operating System: Linux \newline
Document Processing: \LaTeX , LibreOffice \newline
Web Development: HTML, CSS \newline
Others:  Wolfram Mathematica \\

\end{SectionTable}




% --- Section: Additional Qualifications---

\begin{SectionTable}{\headingfont Additional Qualifications}
2014 & 
B.A. in Fine Arts from Sarbabharatiya Sangeet-O-Sanskriti Parishad, Kolkata \\ 
\end{SectionTable}

% --- Section: Outreach Activities, Interests and Public Engagemant---

\begin{SectionTable}{\headingfont Outreach Activities, Interests and Public Engagemant}
& \textbf{Outreach:}\newline Convenor of \href{https://www.ituxitu.in/}{``ituxitu"}, an online \href{https://www.youtube.com/@pbituxitu}{talk series} that connects UG/PG students of Assam to researchers in Physics and Astronomy \newline

\textbf{Quizzing:} \newline Founder and Joint Co-ordinator of Dibrugarh University Debate and Quiz Forum, \newline Represented Dibrugarh University in 50+ quiz competitions at inter-university and state level \\
\end{SectionTable}

\begin{SectionTable}{\headingfont References}
%& \textit{to be produced on demand} \\
& Dr. Debajyoti Dutta \newline
				Assistant Professor \newline
				Department of Physics, Bhattadev University \newline
				\faEnvelope \  \textit{phy.debajyoti@bhattadevuniversity.ac.in} \\
& Dr. Prabwal Jyoti Phukon \newline
				Associate Professor \newline
				Department of Physics, Dibrugarh University \newline
				\faEnvelope \  \textit{prabwal@dibru.ac.in} \\
				

\end{SectionTable}

\begin{SectionTable}{\headingfont Declaration}
& I hereby declare that all the information stated here are correct and complete to the best of my knowledge and belief. \\



& Date: 	6 February, 2025	\newline							          
Place:	 Nalbari, Assam, India \hfill  Himanshu Bora \\
\end{SectionTable}

% --- End of CV! ---

\end{document}





