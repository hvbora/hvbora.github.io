
% --- Set document class and font size ---

\documentclass[a4paper, 11pt]{article}





% --- Package imports ---

\usepackage{hyperref, enumitem, longtable, fontawesome, amsmath, array, graphicx}





% --- Page layout settings ---

% Set page margins
\usepackage[left=0.7in, right=0.8in, bottom=.8in, top=0.8in, headsep=0in, footskip=.2in]{geometry}

% Set line spacing
\renewcommand{\baselinestretch}{1.2}

% --- Page formatting settings ---

% Set link colors
\usepackage[dvipsnames]{xcolor}
\hypersetup{colorlinks=true, linkcolor=MidnightBlue, urlcolor=MidnightBlue}

% Set font to Libertine, including math support
\usepackage{libertine}
\usepackage[libertine]{newtxmath}

% Remove page numbering
\pagenumbering{gobble}

% Define font size and color for section headings
\newcommand{\headingfont}{\Large\color{Peach}}











% --- CV section settings ---

% Note: each section of this table (Education, Awards, Publications etc.) is 
% stored in a two-column table. The left-hand column is narrow (1 inch) and is 
% meant to store dates. The right-hand column is wide (5.2 inches) and stores 
% the main text.  Sections in which each entry might have multiple lines 
% (e.g., Education) are stored in a 'SectionTable' environment). Sections in 
% which each entry might just have one line are stored in a 'SectionTableSingleSpace'
% environment. The only difference between the two environments is the line 
% spacing between each entry. Both environments take one argument, which is the
% title of the section. See main document for how these environments are used.

% Define settings for left-hand column in which dates are printed
\newcolumntype{R}{>{\raggedleft}p{1in}}

% Define 'SectionTable' environment
\newenvironment{SectionTable}[1]{
	\renewcommand*{\arraystretch}{1.7}
	\setlength{\tabcolsep}{10pt}
	\begin{longtable}{Rp{5.2in}} & #1 \\}
{\end{longtable}\vspace{-.3cm}}

% Define 'SectionTableSingleSpace' environment
\newenvironment{SectionTableSingleSpace}[1]{
	\renewcommand*{\arraystretch}{1.2}
	\setlength{\tabcolsep}{10pt}
	\begin{longtable}{Rp{5.2in}} & #1 \\[0.6em]}
{\end{longtable}\vspace{-.3cm}}















% --- Document starts here ---

\begin{document}

% --- Name and contact information ---

\begin{SectionTable}{\Huge \color{Peach} Himanshu Bora}
& 
\faMapMarker \ \ DHSK College $\;\boldsymbol{\cdot}\;$ Dibrugarh, Assam, India \newline
\faEnvelopeO \  \textit{hvbora@gmail.com} \ \ \ \  \faPhone \ +91 84866 09233 \\
%\faGlobe \ www.website.com %
&
{\textbf {Personal Information}} \newline
Date of birth \ \ : \ December 21, 1996 \newline
Sex \ \ \ \ \ \ \ \ \ \ \ \ \ \ \ \ \ : \ Male \newline
Nationality: \ \ \ : \ Indian \newline
Languages: \ \ \ \ : \ English (\textit{Full Professional Working Proficiency}) \newline 
{\color{white} .........................} \ Assamese (\textit{Native}) $\;\boldsymbol{\cdot}\;$ Bengali $\;\boldsymbol{\cdot}\;$ Hindi 
\end{SectionTable}






% --- Section: Education ---
\begin{SectionTable}{\headingfont Education}

2015 -- 2020 & 
\textbf{Integrated M.Sc. in Physics} \newline
Dibrugarh University -- Dibrugarh, Assam, India \newline 
Specialization: High Energy Physics \newline
CGPA: 8.27/10.00 \\


2012 -- 2015 & 
\textbf{HSSLC (12th Standard)} \ Stream: Science \newline
Board: Assam Higher Secondary Education Council (AHSEC) \newline 
Sarupathar H.S. School -- Sarupathar, Assam, India \newline
Precentage: 90.40\% (99.9+ percentile) \\

2012 &
\textbf{HSLC (10th Standard)} \newline
Board: Board of Secondary Education, Assam (SEBA) \newline 
Dayanand Vidya Niketan -- Sarupathar, Assam, India \newline
Precentage: 90.40\% (99.9+ percentile) \\

% --- Un-comment the next few lines if you want to include some courses you've taken ---

%& \textbf{Selected coursework}
%\begin{itemize}[itemsep=0pt, leftmargin=*]
%\item \textit{Statistics}: Asymptotic statistics, Mathematical statistics, Functional data analysis, High-dimensional statistics, Information theory
%\item \textit{Mathematics}: Measure theory, Functional analysis, Measure-theoretic probability %with martingales
%\end{itemize}
\end{SectionTable}






% --- Section: Professional Experience ---

\begin{SectionTable}{\headingfont Professional Expereince}

November 2021 --April 22 & 
\textbf{Lecturer} (\textit{on contract}) \newline
Department of Physics\newline
Dibrugarh Hanumanbax Surajmall Kanoi College -- Dibrugarh, Assam, India  \\
\end{SectionTable}






% --- Section: Research interests ---

%\begin{SectionTable}{\headingfont Research interests}
%& Your favorite topic, another topic, another topic, another topic, another topic
%\end{SectionTable} 





% --- Section: Research experience ---

\begin{SectionTable}{\headingfont Research experience}
October 2020 --July 2021 &
Dibrugarh University -- Dibrugarh, India \newline
\textbf{Topic:} Geometrothermodynamics of CR-BTZ Black Hole \newline
\textbf{Guide:} Dr. Prabwal Jyoti Phukon, Dibrugarh University, Dibrugarh, India  \\

June 2017 --August 2017 &
Bharathiar University -- Coimbatore, India (under FAST-SF program) \newline
\textbf{Title:} A Transformation Method for Construction of Exactly Solvable Potentials \newline
\textbf{Guide:} Prof. S. Saravanan, Bharathiar University, Coimbatore, India
\\
\end{SectionTable}





% --- Section: Publications ---

%\begin{SectionTable}{\headingfont Publications} 
%2021 & 
%\textbf{Title of your most recent research paper} \newline
%First author, second author, third author, fourth author. \newline
%\textit{Journal of something or the other}
%\href{https://en.wikibooks.org/wiki/LaTeX/Hyperlinks}{DOI: }) \\

%\end{SectionTable}






% --- Section: Awards, scholarships, etc ---

\begin{SectionTable}{\headingfont Honors, Awards and Scholarships}
June 2021 & 
Awarded \textbf{INSPIRE Fellowship 2020} provisionally for doctoral research by Department of Science and Technology, Government of India \textit{(provisional award not availed)}
 \\

May 2021 &
Awarded \textbf{Junior Research Fellowship} by Council of Scientific and
Industrial Research, India through \textbf{Joint CSIR-UGC NET Examination, June 2020}
(held in November 2020), \textit{All India Rank-149 (99.51 percentile)} \\

May 2021 &
Obtained eligibility for \textbf{Lectureship / Assistant Professor} in Physical Sciences through \textbf{Joint CSIR-UGC NET Examination, June 2020}
(held in November 2020), \textit{All India Rank-149 (99.51 percentile)} \\

February 2020 &
Qualified \textbf{Graduate Aptitude Test in Engineering-2020 (GATE-2020)} in Physics organised jointly by IISc/IITs (held in February 2020) \\

2017 &
Awarded \textbf{Focus Area Science Technology Summer Fellowship} jointly by Indian Academy of Sciences (Bengaluru); Indian National Science Academy (New Delhi) and The National Academy of Sciences (Prayagraj) \\

2015-2020 (\textit{duration})&
Awarded \textbf{INSPIRE-SHE Scholarship} by Department of Science and Technology, Government of India (\textit{awarded to top 1\% students in India in 10+2, pursuing basic sciences}) \\

2012 &
Awarded \textbf{Anundoram Borooah Award-2012} with distinction by Government of Assam (\textit{awarded to First Division holders in HSLC Examination, 2012 conducted by SEBA, Assam})

\end{SectionTable}










% --- Section: Conferences, Workshops and schools, etc ---

\begin{SectionTable}{\headingfont Schools, Workshops and Conferences Attended}

October 29 -- 30, 2021 & 
Online Workshop on High Performance Computing and AI for Computational Biology by Indian Institute of Technology, Kharagpur and Tezpur University, Tezpur
 \\
October 21 -- 22, 2021 & 
Online Workshop on High Performance Computing in Engineering by Indian Institute of Technology, Kharagpur and Ansys
 \\
 
September 20 --23, 2021 & 
Online Workshop on High Performance Computing for Astrophysics and Astronomy by Square Kilometer Array-India Consortium and Indian Institute of Technology, Kharagpur
 \\

May 10 -- June 11, 2021 &
Introductory Summer School in Astronomy and Astrophysics by IUCAA, Pune, India (\textit{online mode})
\\

August 17 --31, 2021 &
Workshop on Particle Physics, 2020 by Assam Don Bosco University, Guwahati, Assam, India (\textit{online mode})
\\

January 7 --11, 2019 &
Workshop on Celestial Mechanics and Dynamical Astronomy by Central University of Rajasthan, Ajmer, Rajasthan, India
\\

October 7 --13, 2018 &
Workshop on Gravitation and Gravitational Waves by Assam University, Silchar, Assam, India
\\

February 22 -- March 3, 2017 &
ARIES Training School in Observational Astronomy-2017 by Aryabhatta Research Institute of Observational Sciences, Nainital, India
\\

November 2 -- 4, 2016 &
National Workshop on Gravitational Wave Astronomy by Dibrugarh University, Dibrugarh, Assam, India
\\

June 21 -- July 12, 2016 &
CCSU Astronomy and Astrophysics Summer School-2016 by Cotton College State University, Guwahati, Assam, India
\\

November 2 -- 5, 2021 &
National Conference on Current Issues in Cosmology, Astrophysics and High Energy Physics by Dibrugarh University, Dibrugarh, Assam,India
\\
\end{SectionTable}






% --- Section: Talks and tutorials ---

%\begin{SectionTable}{\headingfont Talks and tutorials}
%Month Year &
%Title of your most recent presentation \newline
%\textit{Name of conference, workshop, seminar, etc., or a description} \\

%Month Year &
%Title of your second most recent presentation \newline
%\textit{Name of conference, workshop, seminar, etc., or a description} \\

%Month Year &
%Title of your third most recent presentation \newline
%\textit{Name of conference, workshop, seminar, etc., or a description} %\\
%\end{SectionTable}




% --- Section: Teaching experience ---

%\begin{SectionTable}{\headingfont Teaching experience}
%Fall 2020 & 
%\textbf{Teaching assistant, STAT 123: Course name here (University)} \newline
%Topics and description of your responsibilities. Aliquam volutpat est vel massa. Sed dolor lacus, imperdiet non, ornare non, commodo eu, neque. \newline
%\textit{Average student rating: X/5.} \\
%\end{SectionTable}











% --- Section: Mentorship and service ---

%\begin{SectionTable}{\headingfont Mentorship and service}
%Month Year -- Present &
%\textbf{Title of organization you are in (Name of your role)} \newline
%Description of your responsibilities. Integer pretium semper justo. Proin risus. Nullam id quam. Nam neque. Phasellus at purus et lib ero lacinia dictum. \\






% --- Section: Professional society memberships ---

%\begin{SectionTable}{\headingfont Professional memberships}
%Year -- Present &
%Name of professional society \newline
%\textit{Short description or conferences you attended.} \\

%\end{SectionTable}






% --- Section: Technical Skills ---
\begin{SectionTable}{\headingfont Technical skills}
& \textbf{Languages} \newline
Proficient in: Python \newline
Familiar with: C, C++, Fortran, HTML \\

& \textbf{Softwares} \newline
Document Processing: \LaTeX \newline
Operating Systems: UNIX, Windows \newline
Others: Git, Wolfram Mathematica \\

\end{SectionTable}




% --- Section: Other interests/hobbies ---

\begin{SectionTable}{\headingfont Other Qualifications}
& B.A. in Fine Arts from Sarbabharatiya Sangeet-O-Sanskriti Parishad, Kolkata,2014
\end{SectionTable}



\begin{SectionTable}{\headingfont Other Interests}
& Quizzing: Founder and Joint Co-ordinator of Dibrugarh University Debate and Quiz Forum $\;\boldsymbol{\cdot}\;$ Represented Dibrugarh University in 50+ quiz competitions at university level, district level and state level \newline
Sports: Badminton \newline
Others: Fine Art and Graphic Design
\end{SectionTable}





\begin{SectionTable}{\headingfont References}
& To be produced on demend
\end{SectionTable}

% --- End of CV! ---

\end{document}





